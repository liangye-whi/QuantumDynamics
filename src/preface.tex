% !TEX TS-program = pdflatex
% !TEX root = ../QuantumDynamics.tex

\chapter*{序言}
量子动力学是量子力学的一部分,主要研究抽象量子体系随时间演化的性质,在当今电子转移理论中占有核心地位。作为研究生课程,量子动力学通常包含于量子力学或者高等量子力学等课程中;然而对于量子化学特别是电子转移理论课题组来说,量子动力学完全可以单独作为一门课程开设。

杜克大学化学系的教授Dr. David Beratan是电子转移理论及应用领域的专家,在2017年秋季学期开设了这门量子动力学课程。在课堂上他举了大量的实例,并且使用Mathematica 向学生们做了精妙的展示。然而,一个不可回避的问题是这一课程没有成体系的教科书,主要依赖的阅读材料是将好几本书的部分章节及一些网上的课件拼凑而成的。这也许是美式教学法的一大特点,然而其效率低下的弊端也显露无疑。教科书的缺乏反映了课程的体系性较为松散,而且也造成了巩固知识、预习复习中的诸多困难。本书编者对于这一教学方法难以适应,因此结合自己的学习方式,将其授课内容和阅读材料整合汇总,并且将其中的逻辑理清,编写了这本讲义式教科书。

本书将按照Dr. Beratan讲课的顺序安排。本书不要求使用Mathematica的能力,然而出于实用角度,可能会涉及到一些算法实现方面的考虑。本书默认读者已经具有国内一般本科生量子力学课程的基础,具体内容参考曾谨言老师的《量子力学教程》第三版。

本书编者希望向Dr. Beratan致敬,并声明本书版权与使用权暂归编者所有。

\rightline{梁晔}

\rightline{2017.10}